% -------------------------------------------------------------------
%                  Essential packages (used in template)
% -------------------------------------------------------------------
\usepackage{lipsum} % Zum Erzeugen von Blindtext
\usepackage{uarial}
\renewcommand{\familydefault}{\sfdefault}
\usepackage{blindtext}

\usepackage[utf8]{inputenc}                   % Allow Umlauts (ä, ö, ü)
\usepackage[ngerman]{babel, translator}       % German LaTeX-intern descriptors (Abbildungen, ...)
\usepackage{mathptmx}                         % Font "Times New Roman" in mathematical Environments
\usepackage{courier}                          % Support the use of font "Courier" (used in listings)
\usepackage[T1]{fontenc}                      % Change LaTeX font encoding to support modern fonts
\usepackage{fix-cm}                           % Fix sizes at which CM and EC fonts can be used

\usepackage{geometry}                         % Change text bounds on a per-page basis
\usepackage{fancyhdr}                         % Allow easy customization of headers and footers

\usepackage[ddmmyyyy]{datetime}               % Reformat times from LaTeX commands like \today

\usepackage[usenames,dvipsnames]{xcolor}      % Support text colorization

\usepackage{colortbl}                         % Support applying colors to tables
\usepackage{array}                            % Tables with fixed column size
\usepackage{longtable}                        % Support multi-page tables
\usepackage{multicol}                         % Additional settings for \multicolumn
\usepackage{multirow}                         % Additional settings for \multirow

\usepackage[hyphens,obeyspaces,spaces]{url}   % Allow urls to have line breaks at "-"
\usepackage{hyperref}                         % Clickable hyperlinks in Table of Contents
\usepackage{microtype}                        % Enable "Blocksatz"

\usepackage{float}                            % Support for table H option
\usepackage{floatflt}                         % Support text wrapping around floating environments
\usepackage{wrapfig}                          % Support text wrapping around figures

\usepackage{graphicx}                         % Support including graphics environments for images
\usepackage{caption}                          % Provide greater customization around captions
\usepackage{subcaption}                       % Caption support for subfigures

\usepackage[onehalfspacing]{setspace}         % Set line spacing to be 1.5x the default
\usepackage{listings}                         % Code block support for LaTeX

\usepackage[titles]{tocloft}                  % More options to modify list of figures & tables
\usepackage{nomencl}                          % Support for nomenclatures ("Symbolverzeichnis")

\usepackage{amsmath}                          % Includes a plethora of mathematical packages 
\usepackage{amssymb}                          % Source mathematical symbols like \sin
\usepackage{amsthm}                           % Better theorems
\usepackage{upgreek}                          % Use \Upomega to get a straight \Omega
\usepackage{siunitx}                          % Unified support for si units

\usepackage{ifthen}                           % Support better if statements in LaTeX


% Support list of acronyms and glossary
\usepackage[
  automake,
  toc,
  nogroupskip,
  nonumberlist,
  nopostdot,
  acronyms,
  shortcuts,
  translate=babel
]{glossaries}

% Track the total count of different environments
% Needs to be imported because LaTeX resets these counter on a chapter basis
\usepackage[figure,table,lstlisting,xspace]{totalcount}

% Annotate different information directly into the document
\setlength{\marginparwidth}{2cm}
\usepackage[
  colorinlistoftodos,
  prependcaption,
  textsize=tiny
]{todonotes}

% -------------------------------------------------------------------
%                 Useful packages (not used in template)
% -------------------------------------------------------------------

\usepackage{pdfpages}                         % Include existing PDFs into the work

% Create electrical diagrams within LaTeX
\usepackage[
  european,
  oldvoltagedirection,
  straightvoltages,
  siunitx
]{circuitikz}